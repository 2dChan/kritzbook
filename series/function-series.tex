\documentclass[12pt, oneside]{article}
\usepackage[T2A, T1]{fontenc}
\usepackage[utf8]{inputenc}
\usepackage[english, russian]{babel}
\usepackage{amssymb, amsthm, amsmath}
\usepackage[pdftex]{graphicx}
\usepackage{pgfplots}
\pgfplotsset{compat=1.18}

\theoremstyle{plain}

\textwidth=170mm
\textheight=250mm
\hoffset=-20mm
\voffset=-30mm

\parindent=0.5cm
\parskip=0.1cm

\tolerance=400

\binoppenalty=10000
\relpenalty=10000
\sloppy

\newtheorem{Lemma}{Лемма}
\newtheorem{Proposition}{Утверждение}
\newtheorem{Theorem}{Теорема}
\newtheorem{Corollary}{Следствие}
\newtheorem{Definition}{Определение}
\newtheorem{Problem}{Задача}
\newtheorem{Example}{Пример}
\newtheorem{Note}{Замечание}
\newtheorem{T}{Обозначение}
\renewcommand{\proofname}{Доказательство}

\DeclareMathOperator*{\lowlim}{\underline{lim}}
\DeclareMathOperator*{\uplim}{\overline{lim}}

\newlength{\lenun}
\newlength{\lendu}
\settowidth{\lenun}{$W$}
\settoheight{\lendu}{$W$}
\def\Wo{\makebox{\parbox[c][\lendu][b]{\lenun}{$\stackrel{o}{W}$}}}

\makeatletter
\renewcommand{\@listI}{%
\leftmargin=25pt \rightmargin=0pt \labelsep=5pt \labelwidth=20pt \itemindent=0pt
\listparindent=0pt \topsep=8pt plus 2pt minus 4pt \partopsep=2pt plus 1pt minus
1pt \parsep=0pt plus 1pt \itemsep=\parsep}
\makeatother

\begin{document}
\begin{minipage}{0.25\textwidth}
	\begin{equation}\nonumber
	\begin{split}
		& \{f_n{x}\}_{n=1}^\infty \\
		& \sum_{n=1}^\infty u_n(x)
	\end{split}
	\end{equation}
\end{minipage}
\hfill
\begin{minipage}{0.65\textwidth}
	Будем предпологать, что все \(f_n(x)\) / все \(u_n(x)\)
	являются функциями, определенными на некотором общем множестве \(X C \mathbb{R}\)
\end{minipage}
\section{Различные виды схоимости}
Пусть \(x_0 \in \mathbb{X}: \{f_n(x_0)\} \to_{n\to\infty} f(x_0)\) - Поточечная
сходимость.

\begin{equation}\nonumber
	\sum_{n=1}^{\infty} u_n(x_0)\ \text{сходится к}\ S(x_0) \text{- Поточечная сходимость.}
\end{equation}
\begin{Example} 
	В разработке.
\end{Example}
\begin{Example}
	В разработке.
\end{Example}

\begin{Definition}
	\(\{f_n(x)\}\) равномерно на \(X\) сходится к \(f(x)\), если
	\[
		\forall \varepsilon > 0 \exists N : \forall n \geq N \forall x \in X: 
		\left|f_n(x) - f(x)\right| < \varepsilon
	\]
\end{Definition}
\begin{Definition}
	\(\sum_{n=1}^{\infty}u_n()x\) называется равномерно сходящимся на X, если
	посл-ть \(\{S_n(x)\}\) его част. сумм. равномерно сходятся на X.
\end{Definition}
\begin{Note}
	Равномерная сходимость на X подразумевает сходимость в \(\forall\) точке
	\(x_0 \in X\) Из поточечной сходимости, вообще говоря, не следует равномерная
	сходимость.
	Из примера 2: \(\{x^n\}\to 0 \forall x \in (-1, 1)\)
\end{Note}
\begin{proof}
	Докажем, что на \((-1, 1)\) нет равномерной сходимости:
	\begin{equation} \nonumber
	\begin{split}
	  \exists \varepsilon \geq 0 \forall N \ & \exists n \geq N \exists x \in (-1, 1): |x^n| \geq 
		\varepsilon\\
																		& n = N \quad x = 1 - \frac{1}{N} \Rightarrow 
																		(1 - \frac{1}{N}) \to \frac{1}{e} \Rightarrow
																		\text{При} N \geq N_0 : \left(1 - \frac{1}{N}
																		\right)^N \geq \frac{1}{3}
	\end{split}
	\end{equation}
\end{proof}
\begin{Note}
	Из равномерное схожимости на X следует равномерная сходимость на \(\forall\)
	подмножестве \(X_1 c X\)
\end{Note}
\begin{Note}
	Равномерная сходимость \(\{f_n(x)\}\) на \(X \text{к} f(x)\) равносильная
	\[
		\exists lim_{n\to\infty} \sup_{x \in X} |f_n(x) - f(x)| = 0
	\]
\end{Note}

\begin{T}
	\begin{equation} \nonumber
	\begin{split}
		\{f_n(x)\} &\to^{\to^X} f(x) \\
		\sum_{n=1}^{\infty} u_n(x) &\to^{\to^X} S(x)
	\end{split}
	\end{equation}
\end{T}

\begin{Theorem}[Критерий Коши] \
	\begin{equation} \nonumber
		\begin{split}
			& I. \{f_n(x)\} \to^{\to} \text{на} X \Leftrightarrow \forall \varepsilon > 0
			\exists N \forall n \geq N \forall p \in \mathbb{N} \forall x \in \mathbb{X}:
			\left|f_{n+p}(x) -f_n(x)\right| < \varepsilon \\
			& II. \sum_{n=1}^{\infty} u_n(x) \to^{\to} \text{на} X \Leftrightarrow \forall \varepsilon > 0
			\exists N \forall n \geq N \forall p \in \mathbb{N} \forall x \in \mathbb{X}:
			\left|\sum_{n=1}^{\infty} u_n(x) \right| < \varepsilon
		\end{split}
	\end{equation}
\end{Theorem}

\begin{proof}
	В разработке.
\end{proof}
\begin{Corollary}
	В разработке.
\end{Corollary}

\end{document}
