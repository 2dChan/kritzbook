\documentclass[12pt, oneside]{article}
\usepackage[T2A, T1]{fontenc}
\usepackage[utf8]{inputenc}
\usepackage[english, russian]{babel}
\usepackage{amssymb, amsthm, amsmath}
\usepackage[pdftex]{graphicx}
\usepackage{pgfplots}
\pgfplotsset{compat=1.18}

\theoremstyle{plain}

\textwidth=170mm
\textheight=250mm
\hoffset=-20mm
\voffset=-30mm

\parindent=0.5cm
\parskip=0.1cm

\tolerance=400

\binoppenalty=10000
\relpenalty=10000
\sloppy

\newtheorem{Lemma}{Лемма}
\newtheorem{Proposition}{Утверждение}
\newtheorem{Theorem}{Теорема}
\newtheorem{Corollary}{Следствие}
\newtheorem{Definition}{Определение}
\newtheorem{Problem}{Задача}
\newtheorem{Example}{Пример}
\newtheorem{Note}{Замечание}
\renewcommand{\proofname}{Доказательство}

\DeclareMathOperator*{\lowlim}{\underline{lim}}
\DeclareMathOperator*{\uplim}{\overline{lim}}

\newlength{\lenun}
\newlength{\lendu}
\settowidth{\lenun}{$W$}
\settoheight{\lendu}{$W$}
\def\Wo{\makebox{\parbox[c][\lendu][b]{\lenun}{$\stackrel{o}{W}$}}}

\makeatletter
\renewcommand{\@listI}{%
\leftmargin=25pt \rightmargin=0pt \labelsep=5pt \labelwidth=20pt \itemindent=0pt
\listparindent=0pt \topsep=8pt plus 2pt minus 4pt \partopsep=2pt plus 1pt minus
1pt \parsep=0pt plus 1pt \itemsep=\parsep}
\makeatother

\begin{document}
	\begin{Theorem}
		Если хотя бы один из 4 рядов сходится абсолютно, то абсолютно сходятся и все 
		остальные ряды, причем к той же сумме.
	
	\begin{equation}
	\begin{split} \nonumber
		& \sum_{m,n=1}^{\infty} a_{mn} ~ S_1 \\
		& \sum_{m = 1}^{\infty}\sum_{n=1}^{\infty} a_{mn} ~ S_2 \\
		& \sum_{n = 1}^{\infty}\sum_{m=1}^{\infty} a_{mn} ~ S_3 \\
		& \sum_{r=1}^{\infty} a'_r ~ S_4
	\end{split}
	\end{equation}
	\end{Theorem}

	\begin{proof}
		1. \(a_mn \geq 0 \forall m, n\) \\
		Сходимость 1го \(\Rightarrow\) Сходимость 2го и \(S_2 = S_1\) \\
		Воспользуемся теормой 22: т.к все строчные ряды\\
		\(\lsum_{n=1}^{\infty} a_mn\) таковы, что \(\lsum_{n=1}^{N} a_mn \leq 
		\lsum_{k=1}^{m}\lsum_{n=1}^{N}a_mn \leq const\), а следовательно, они 
		сходятся. Таким образом 2й ряд сходится к \(S_1\).\\
		Сходимость 2го ряда \(\Rightarrow\) Сходимость 1го рода и \(S_1 = S_2\)
		\begin{equation} \nonumber
			\sum_{m=1}^{M}\left(\sum_{n=1}^{N} a_mn\right) \leq
			\sum_{m=1}^{M}\sum_{n=1}^{\infty} a_mn \leq S_2
			\Rightarrow \textbf{По теореме 22} \Rightarrow S_2 = S_1
		\end{equation}
		Сходимость 1го \(\Rightarrow\) Сходимость 4го ряда и \(S_4 = S-1\)
		\begin{equation} \nonumber
			\sum_{r=1}^{R} a'_r
		\end{equation}
		Среди \(a'_1, ..., a'_2\) найдется такие M,N, которые являются max номера 
		строки и номера столбца, где они расположены в \(\{a_mn\}\)

		\begin{equation} \nonumber
		\begin{split}
			& \sum_{r=1}^{R} a'_r \leq \sum_{m=1}^{M}\sum_{n=1}^{N} a_mn \leq const
			\Rightarrow \text{4й ряд сходится к} S_4 \\
			& \text{Из} \leq \Rightarrow S_4 \leq S_1
		\end{split}
		\end{equation}

		Докажем, что \(S_1 \leq S_4\)
		\begin{equation}\nonumber
			\sum_{m=1}^{M}\sum_{n=1}^{N} a_mn \leq \sum_{r=1}^{R} a'_r \leq S_4
		\end{equation}
		\(\exists\)
		Дописать.
	\end{proof}

	2. \(a_mn \in \mathbb{R}: \)
	%\begin{equation} \nonumber
	%\begin{split}
	%	&	p_mn = \frac{a_mn + |a_mn|}{2}, q_mn = \frac{-a_mn + |a_mn|}{2} \\
	%	& 0 \leq p_mn \leq |a_mn| \qquad p_mn -q_mn = a_mn \\
	%	& 0 \leq q_mn \leq |a_mn|
	%\end{split}
	%\end{equation}
	Докажем, например абсолюьную сходимость 2го ряда \(\Rightarrow\) абсолютная
	сходимость 1го и \(S_1 = S_2\)\\
	%\begin{equation}
	%	\sum_{m=1}^{\infty} \Rightarrow 
	%	\left\{ \begin{aligned}
	%			f(x) &=
	%	\end{aligned} \right.
	%\end{equation}


\end{document}
