\documentclass[12pt, oneside]{article}
\usepackage[T2A, T1]{fontenc}
\usepackage[utf8]{inputenc}
\usepackage[english, russian]{babel}
\usepackage{amssymb, amsthm, amsmath}
\usepackage[pdftex]{graphicx}
\usepackage{pgfplots}
\pgfplotsset{compat=1.18}

\theoremstyle{plain}

\textwidth=170mm
\textheight=250mm
\hoffset=-20mm
\voffset=-30mm

\parindent=0.5cm
\parskip=0.1cm

\tolerance=400

\binoppenalty=10000
\relpenalty=10000
\sloppy

\newtheorem{Lemma}{Лемма}
\newtheorem{Proposition}{Утверждение}
\newtheorem{Theorem}{Теорема}
\newtheorem{Corollary}{Следствие}
\newtheorem{Definition}{Определение}
\newtheorem{Problem}{Задача}
\newtheorem{Example}{Пример}
\newtheorem{Note}{Замечание}
\renewcommand{\proofname}{Доказательство}

\DeclareMathOperator*{\lowlim}{\underline{lim}}
\DeclareMathOperator*{\uplim}{\overline{lim}}

\newlength{\lenun}
\newlength{\lendu}
\settowidth{\lenun}{$W$}
\settoheight{\lendu}{$W$}
\def\Wo{\makebox{\parbox[c][\lendu][b]{\lenun}{$\stackrel{o}{W}$}}}

\makeatletter
\renewcommand{\@listI}{%
\leftmargin=25pt \rightmargin=0pt \labelsep=5pt \labelwidth=20pt \itemindent=0pt
\listparindent=0pt \topsep=8pt plus 2pt minus 4pt \partopsep=2pt plus 1pt minus
1pt \parsep=0pt plus 1pt \itemsep=\parsep}
\makeatother

\begin{document}
	\section{Бесконечные произведения}
	\[
		v_{1} * v_{2} * ... *v_{n} * ... = \prod_{n = 1}^{\infty}v_{n} \qquad(1)
	\]
	\begin{Definition}
		Бесконечное произведение (1) - сходится к числу $P \neq 0$, если
		$P_{n} = \prod\limits_{k = 1}^{n}v_{k} \to P\ ({n\to\infty})$. Если $\exists
		\lim\limits_{n\to\infty}P_{n} = 0$, то говорят, что бесконечное произведение
		(1) расходится к 0. \\
		Если $\nexists \lim\limits_{n\to\infty}P_{n}$, то (1) расходится.\\
		Заметим что $v_{n} = \frac{P_{n}}{P_{n-1}}$. 
		Если $\exists P \neq 0 : P_{n} \to P$ то $v_{n} \to 1$
	\end{Definition}

	\begin{Theorem}
		{Необходимое условие сходимости} Если бесконечное произведение $(1)$ сходится,
		то ${v_n}\to 1 (n \to \infty)$.
	\end{Theorem}
	\begin{proof}
		Т.к отбрасывание конечного числ $v_{n}$ не влияет на его сходимость и,
		начиная с некоторого $n_{0}: v_{n} >= \frac{1}{2}(n \geq n_{0})$, можно считать,
		что в бесконечных произведениях все $v_{n} > 0$.
	\end{proof}
	\begin{Example}
		\[
			\sqrt{\frac{1}{2}}*\sqrt{\frac{1}{2} + \frac{1}{2} \sqrt{\frac{1}{2}}}* \sqrt{\frac{1}{2}
			+ \frac{1}{2}\sqrt{\frac{1}{2} + \frac{1}{2}\sqrt{\frac{1}{2}}}}* \sqrt{\frac{1}{2}
			+ \frac{1}{2} * (...)}* ... = \frac{2}{\pi}
		\]
		\[
			sinx = 2sin\frac{x}{2}cos\frac{x}{2}= 4 sin\frac{x}{4}cos\frac{x}{4}cos\frac{x}{2}
			= ... = 2sin\frac{x}{2^{n}}cos\frac{x}{2^{n}}cos\frac{x}{2^{n-1}}*...*cos\frac{x}{2}
		\]
		$\prod\limits_{k = 1}^{\infty}cos\frac{x}{2^{k}}\Rightarrow$ частичное произведение
		$P_{n} = \frac{sinx}{2^{n}sin\frac{x}{2^n}}= \frac{sinx}{x}* \frac{\frac{x}{2^{n}}}{sin\frac{x}{2^n}}
		\to \frac{sinx}{x}$

		$\prod\limits_{k = 1}^{\infty}cos\frac{x}{2^{k}}= \frac{sinx}{x}\Rightarrow \prod
		\limits_{k = 1}^{\infty}cos\frac{x}{2^{k+1}}= \frac{\pi}{2}\Rightarrow$
		формула Виета
	\end{Example}
	\begin{Example}
		Валиис
		\begin{equation}
		\begin{split}
			I_{n} 
			&= \int_{0}^{\frac{\pi}{2}}sin^{n}xdx = 
			  \int_{0}^{\frac{\pi}{2}}sin^{n-1}x d(-cosx) = -sin^{n-1}{x}*cosx +\\ 
			&+ \int_{0}^{\frac{\pi}{2}}cosx * (n-1)sin^{n-2}x\cos{x}dx =
			  (n - 1) \int_{0}^{\frac{\pi}{2}}(1-sin^{2}x)*sin^{n-2}xdx =\\
			&= (n - 1)I_{n-2}-(n -1)I_{n} \Rightarrow I_{n} = \frac{n - 1}{n}I_{n-2}\\
		  I_{2k} 
      &= \frac{2k-1}{2k}I_{2k-2}=\frac{(2k-1)(2k-3)}{2k(2k-2)}I_{2k-4} = 
			  \frac{(2k- 1)(2k - 3)...1}{2k(2k-2)...2}I_{0} \\
			I_{2k + 1}
		  &= \frac{2k}{2k+1}I_{2k-1}=\frac{(2k)(2k-2)}{(2k+1)(2k-1)}I_{2k-3} = \frac{2k(2k - 2)...2}{(2k+1)(2k-1)...3}I_{1} \\
			\frac{I_{2k}}{I_{2k + 1}}
			&= \frac{(2k + 1)(2k - 1)^{2}(2k - 3)^{2}*...*3^{2}}{(2k)^{2}(2k-2)^{2}*...*2^{2}} * 
			  \frac{\pi}{2}\Rightarrow \frac{\pi}{2} = \frac{(2k + 1)(2k - 1)^{2}*...*3^{2}}{(2k)^{2}(2k-2)^{2}*...*2^{2}} * \frac{I_{2k}}{I_{2k + 1}}\nonumber
		\end{split}
		\end{equation}

		Докажем, что $\frac{I_{2k}}{I_{2k+1}}\to 1 (k \to \infty)$
		\[
			1 < \frac{I_{2k}}{I_{2k+1}}< \frac{I_{2k-1}}{I_{2k + 1}}= \frac{2k + 1}{2k}
			\to 1
		\]
	\end{Example}
	Так как $\ln{P_n}= \ln{v_1}+ ... + \ln{v_n}\\
	\sum\limits_{n = 1}^{\infty}ln(v_{n})$

	\begin{Theorem}
		Пусть $v_{n} > 0 \forall n$ Тогда бесконечное произведение(1) сходится
		$\Leftrightarrow$ сходится ряд $\sum\limits_{n = 1}^{\infty}\ln{v_n}$
	\end{Theorem}

	\begin{Theorem}
		Пусть $v_{n} \geq 1 \forall n$ Тогда бесконечное произведение(1) сходится
		$\Leftrightarrow$ сходится ряд $\sum\limits_{n = 1}^{\infty}(v_{n} - 1)$
	\end{Theorem}
	\begin{proof}
		$\Rightarrow$ {т.17} $v_{n} \to 1$ Для ряда $\sum\limits_{n = 1}{\infty}\ln{v_n}$
		из неотриц. слагаемых$(v_{n} \geq 1)$ выполнено $\lim\limits_{n\to\infty}\frac{\ln{v_n}}{v_{n}-1}
		= 1$ По пр-ку сравнения сход. ряд. $\sum\limits_{n=1}{\infty}(v_{n} - 1)$ \\
		$\Leftarrow$ Если ряд $\sum\limits_{n = 1}^{\infty}(v_{n} - 1)$ сх.
		$\Rightarrow v_{n} \to 1$ и
		$\lim\limits_{n \to \infty}\frac{\ln{v_n}}{v_{n} - 1}= 1 \Rightarrow$ по
		признаку сравнения $\Rightarrow$ сходится ряд
		$\sum\limits_{n = 1}^{\infty}\ln(v_{n})$
	\end{proof}
	\begin{Theorem}
		Пусть $v_{n} > 0 \forall n$ Положим $v_{n} = 1 + u_{n} \forall n$
		\begin{enumerate}
			\item Пусть ряд $\sum\limits_{n=1}^{\infty}u_{n}$ сходится. Тогда
				бесконечное произведение сходится $\Leftrightarrow$ ряд
				$\sum\limits_{n = 1}^{\infty}u_{n}^{2}$ сзрдится.

			\item Пусть ряд $\sum\limits_{n = 1}^{\infty}u^{2}_{n}$ сходится. Тогда
				бесконечное произведение (1) сходится $\Leftrightarrow$ ряд
				$\sum\limits_{n = 1}^{\infty}u_{n}$ сходится.
		\end{enumerate}
	\end{Theorem}
	\begin{proof}
		В условиях т. 20: $u_{n} \to 0, n \to \infty \ \sum\limits_{n = 1}^{\infty}\ln
		{1 + u_n}$
		\[
			\ln{1 + u_n}= u_{n} - \frac{u_{n}}{2}+ o(u^{2}_{n}), n\to\infty
		\]
	\end{proof}
\end{document}
