\documentclass[12pt, oneside]{article}
\usepackage[T2A, T1]{fontenc}
\usepackage[utf8]{inputenc}
\usepackage[english, russian]{babel}
\usepackage{amssymb, amsthm, amsmath}
\usepackage[pdftex]{graphicx}
\usepackage{pgfplots}
\graphicspath{ {../images/} }
\pgfplotsset{compat=1.18}

\theoremstyle{plain}

\textwidth=170mm
\textheight=250mm
\hoffset=-20mm
\voffset=-30mm

\parindent=0.5cm
\parskip=0.1cm

\tolerance=400

\binoppenalty=10000
\relpenalty=10000
\sloppy

\newtheorem{Lemma}{Лемма}
\newtheorem{Proposition}{Утверждение}
\newtheorem{Theorem}{Теорема}
\newtheorem{Corollary}{Следствие}
\newtheorem{Definition}{Определение}
\newtheorem{Problem}{Задача}
\newtheorem{Example}{Пример}
\newtheorem{Note}{Замечание}
\renewcommand{\proofname}{Доказательство}

\newcommand{\llim}{\lim\limits}
\newcommand{\lsum}{\sum\limits}

\DeclareMathOperator*{\lowlim}{\underline{lim}}
\DeclareMathOperator*{\uplim}{\overline{lim}}

\newlength{\lenun}
\newlength{\lendu}
\settowidth{\lenun}{$W$}
\settoheight{\lendu}{$W$}
\def\Wo{\makebox{\parbox[c][\lendu][b]{\lenun}{$\stackrel{o}{W}$}}}

\makeatletter
\renewcommand{\@listI}{%
\leftmargin=25pt \rightmargin=0pt \labelsep=5pt \labelwidth=20pt \itemindent=0pt
\listparindent=0pt \topsep=8pt plus 2pt minus 4pt \partopsep=2pt plus 1pt minus
1pt \parsep=0pt plus 1pt \itemsep=\parsep}
\makeatother

\begin{document}
	\section{Понятие числовых рядов}
	\begin{equation} \tag{1}
		\{ u_n \}_{n \geq 1} \subset \mathbb{R} : u_1 + u_2 + ... + u_n + ... \equiv 
		\sum_{n = 1}^{\infty} u_n \qquad 
	\end{equation}

	\begin{Definition}
		Числовой ряд (1) сходится к числу S, называется его суммой, если 
		\(\exists \llim_{N\to\infty} \lsum_{n=1}^{N} u_n = S \). \\
		(N-ая) частичная сумма: \(S_N = \lsum_{n=1}^{N} u_n\)
	\end{Definition}
	\begin{Definition}
		Если \(\llim_{N\to\infty}S_N\) не существует, то говорят, что ряд (1) расходится.
	\end{Definition}

	\begin{Example}
		%TODO: сделать.
		в разработке.
	\end{Example}

	\begin{Theorem}[Критерий Коши] \

		Числовой ряд сходится \(
			\Leftrightarrow \forall \varepsilon > 0 \ \exists N : 
			\forall n \geq N, p \in \mathbb{N} : \left|\lsum_{k=N+1}^{N+p} u_k \right| < 
			\varepsilon 
		\)
	\end{Theorem}
	
	\begin{proof}
		Следует из опр и критерия Коши сходимости последвательности.
		\begin{equation}
		\begin{split}
			|S_{N +p} - S_N| =
			\left| \sum_{n = 1}^{N + p} u_n - \sum_{n = 1}^{N} u_n \right| = 
			\left|\sum_{n=N+1}^{N+p} u_n \right| < \varepsilon
			\nonumber
		\end{split}
		\end{equation}
	\end{proof}
	\begin{Corollary}
		Если ряд (1) сходится, то \(\llim_{n\to\infty} u_n = 0\)
	\end{Corollary}
	\begin{Example} \
		\begin{itemize} 
			\item \(1 + \frac{1}{2} + \frac{1}{3} + ... + \frac{1}{n} + ... \) 
				гармонический ряд расходится.
			\item \(\lsum_{n=1}^{\infty} \frac{1}{sin^2 n * n^3}\) - расходится.

			\item \(\lsum_{n = 1}^{\infty} \frac{1}{n^2} = \frac{\pi^2}{6}\)
	\end{itemize}
	\end{Example}
	\begin{Note}
		Отбрасывание любого конечного числа слагаемых не влияет на сходимость.
	\end{Note}
	\begin{Note}  \
		\begin{itemize}
			\item \(\lsum_{n=1}^{\infty} u_n \pm \lsum_{n=1}^{\infty} v_n =
				\lsum_{n=1}^{\infty} (u_n \pm v_n) \) - 
				сложение двух сходящихся рядов дают сходящийся ряд.
			\item \(\alpha(\lsum_{n=1}^{\infty} u_n) = \lsum_{n=1}^{\infty} \alpha u_n\) - умножение ряда на число дает сходящийся ряд.
		\end{itemize}
	\end{Note}
\end{document}
